\section{HashCode-Methode / Hashing}
\subsection{HashCode-Methode}
\begin{itemize}
    \item Muss überschrieben werden, wenn \lstinline{equals()} überschrieben wird.
    \item Gibt einen Hashcode zurück, der für gleiche Objekte gleich ist.
    \item Sollte möglichst effizient sein.
\end{itemize}

\subsection{Syntax}

\lstinputlisting{snippets/hashcode.java}
Bei mehreren Instanzvariablen: Hashcodes der Instanzvariablen mit verschiedenen Primzahlen multiplizieren und addieren.

\subsection{Hashing}
\begin{itemize}
    \item Elemente werden auf Array (Hash-Tabelle) verstreut.
    \item Hash-Code wird aus Objekt berechnet und bestimmt den Index.
\end{itemize}

\subsubsection{Regeln}
\mylstbox{x.equals(y)}\textrightarrow\mylstbox{x.hashCode() == y.hashCode()}
\begin{itemize}
    \item Umkehrung gilt nicht notwendigerweise.
    \item Überschreibung von \lstinline{equals()} erfordert Überschreibung von \lstinline{hashCode()}.
\end{itemize}

\subsubsection{Beispiel}
\begin{tikzpicture}
    \begin{scope}[every node/.style={text=white, minimum width=1.5cm, minimum height=0.8cm, rounded corners=2mm, align=center}]
        \node[fill=redcontrast] (abc) at (0,0) {Message\\"ABC"};
        \node[fill=basegreen, below=3mm of abc] (def) {Message\\"DEF"};        
    \end{scope}

    \begin{scope}[every node/.style={single arrow, draw, minimum height=7mm, single arrow head extend=1mm}]
        \node[right=2mm of abc.east] (arr1) {};
        \node[right=2mm of def.east] (arr2) {};        
    \end{scope}

    \node[right=0mm of arr1.east] (hc1) {\lstinline{hashCode() == 4}};
    \node[right=0mm of arr2.east] (hc2) {\lstinline{hashCode() == 2}};

    \matrix[row sep = -0.1mm,matrix anchor=idx3.west, nodes={draw, minimum width=1.5cm, minimum height=0.4cm}, right=1cm of hc2.east]
    {
        \node (idx0) {Index 0}; \\
        \node {}; \\
        \node (idx3) {}; \\
        \node {}; \\
        \node (idx5) {}; \\
    };
    \node[above=0mm of idx0, align=center] {Hash-Tabelle};

    \begin{scope}[>=Stealth]
        \draw[->] (hc1.east) -- (idx5.west);
        \draw[->] (hc2.east) -- (idx3.west);
    \end{scope}

    \begin{scope}[every node/.style={text=white, minimum width=1.5cm, minimum height=0.8cm, rounded corners=2mm, align=center}]
        \node[fill=redcontrast, right=5mm of idx5.east] (abc2) {Message\\"ABC"};
        \node[fill=basegreen, right=5mm of idx3.east] (def2) {Message\\"DEF"};        
    \end{scope}
    \begin{scope}[cbase /.tip={Arc Barb[reversed, line width=1pt, width=2pt, length=1pt]},>=Stealth]
        \draw[{cbase}->] (idx5.east) -- (abc2.west);
        \draw[{cbase}->] (idx3.east) -- (def2.west);
    \end{scope}
\end{tikzpicture}