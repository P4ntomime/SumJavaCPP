\section{Exceptions}

\begin{minipage}[t]{0.6\columnwidth}
    \subsection{Checked Exceptions}
    \raggedright%
    Bei Methodendeklaration müssen Exceptions angegeben werden, welche gechecked werden müssen.
    \lstinputlisting{snippets/exceptchecked.java}
    Es können nach \lstinline{throws} auch mehrere Exceptions mit Komma separiert angegeben werden.
\end{minipage}\hfill%
\begin{minipage}[t]{0.4\columnwidth}
    \subsection{Unchecked Exceptions}
    \raggedright%
    \begin{itemize}
        \item Keine \lstinline{throws}-Deklaration und keine Behandlung nötig
        \item \lstinline{RuntimeException} und \lstinline{Error} sowie Subklassen
        \item Behebung nur durch Änderung des Codes
    \end{itemize}
\end{minipage}

\vspace{-1.3cm}
\begin{minipage}[t]{0.65\columnwidth}
    \subsection{Exceptions behandeln}
    Exceptions werden mit \lstinline{try-catch}-Blöcken behandelt.
    \lstinputlisting{snippets/exceptcatch.java}
\end{minipage}\hfill%
\raisebox{1.4cm}{\begin{minipage}[t]{0.325\columnwidth}
    \subsection{Ablauf}
    \subsubsection{Normalfall}
    \begin{tikzpicture}[every node/.style={draw, 
        fill=bluecontrast, 
        text=white, 
        text width=1.4cm, 
        minimum height=3mm,
        inner sep=0mm, 
        text depth=3mm,
        text height=2mm,
        align=flush left}, >={Stealth[length=1.25mm]}]
        \node at (0,-0.5) (try) {try};
        \node[fill=redcontrast,anchor=north west, postaction={draw,thin}, line width=0mm] at ($(try.south)!0.5!(try.south east)$) (catch) {catch};
        \draw[->] (0,0) -- (try.north);
        \draw[white, ->] (try.north) -- (try.south);
        \draw[lightgray, dash pattern=on 1.5pt off 1.5pt, ->] (try.south) -- (try.south |- catch.south);
        \draw[->] (try.south |- catch.south) -- +(0,-0.25);
        \draw[coralcontrast,-] ($(catch.south west)+(-0.2pt,0mm)$) -- (try.south west |- catch.south);
    \end{tikzpicture}

    \subsubsection{Ausnahmefall}
    \begin{tikzpicture}[>={Stealth[length=1.25mm]}]
        \begin{scope}[every node/.style={draw, 
            fill=bluecontrast, 
            text=white, 
            text width=1.4cm, 
            minimum height=3mm,
            inner sep=0mm, 
            text depth=3mm,
            text height=2mm,
            align=flush left}]
            \node at (0,-0.5) (try) {try};
            \node[fill=redcontrast,anchor=north west, postaction={draw,thin}, line width=0mm] at ($(try.south)!0.5!(try.south east)$) (catch) {catch};
        \end{scope}
        \node[fill=claretcontrast, star, star point height=1.235mm, minimum size=4mm, inner sep=0mm, star points=5, draw=lightgray, anchor=inner point 2, xshift=0.7mm] (someexcept) at (try.center) {};
        \draw[lightgray, dash pattern=on 1.5pt off 1.5pt, ->, looseness=0.5] (someexcept) to[out=0,in=90] (catch.north);
        \draw[->] (0,0) -- (try.north);
        \draw[white, ->] (try.north) -- (try.north |- someexcept.inner point 1);
        \draw[white, ->] (catch.north) -- (catch.south);
        \draw[->] (try.south |- catch.south) -- +(0,-0.25);
        \draw[coralcontrast,-] ($(catch.south west)+(-0.2pt,0mm)$) -- (try.south west |- catch.south);

    \end{tikzpicture}
\end{minipage}}

\vspace{-2ex}
\subsection{Stack Unwinding}
\begin{itemize}
    \item Baue solange Methoden-Frames vom Stack ab, bis Exception behandelt wird
    \item Falls \lstinline{main()} mit Exception zurückkehrt, behandelt JVM Exception mit Programmabbruch
\end{itemize}

\subsection{Error vs. Exception}
\vspace{-.9\abovedisplayskip}
\begin{minipage}[t]{0.5\columnwidth}
    \subsubsection{Error}
    \raggedright%
    \begin{itemize}
        \item Schwerwiegende Fehler, die \textbf{nicht behandelt} werden sollen
        \item Fehler in JVM: \lstinline{OutOfMemoryError}, \lstinline{StackOverflowError} % ChkTeX 13
        \item Programmierfehler: \lstinline{AssertionError}
    \end{itemize}
\end{minipage}\hfill%
\begin{minipage}[t]{0.49\columnwidth}
    \subsubsection{Exception}
    \raggedright%
    \begin{itemize}
        \item Laufzeitfehler, die \textbf{behandelbar} sind
        \item Fehler in Eingabe, Parameter, Bedienung, \ldots z.B. \lstinline{IOException} \textrightarrow{} grundsätzlich behandeln
        \item Andere Laufzeitfehler \textrightarrow{} lieber nicht behandeln
    \end{itemize}
\end{minipage}
    
\subsection{\textsf{\textbf{\textcolor{keywordcolour}{finally}}}-Ablauf}
\vspace{-.9\abovedisplayskip}
\begin{minipage}[t]{0.33\columnwidth}
    \subsubsection{Normalfall}
    \begin{tikzpicture}[every node/.style={draw, 
        fill=bluecontrast, 
        text=white, 
        text width=1.6cm, 
        minimum height=3mm,
        inner sep=0mm, 
        text depth=3mm,
        text height=2mm,
        align=flush left}, >={Stealth[length=1.25mm]}]
        \node at (0,-0.5) (try) {try};
        \node[fill=redcontrast,anchor=north west, postaction={draw,thin}, line width=0mm] at ($(try.south)!0.5!(try.south east)$) (catch) {catch};
        \node[fill=basegreen, anchor=north east, postaction={draw,thin}, line width=0mm] at ($(catch.south)!0.5!(catch.south west)$) (finally) {finally};
        \draw[->] (0,0) -- (try.north);
        \draw[white, ->] (try.north) -- (try.south);
        \draw[lightgray, dash pattern=on 1.5pt off 1.5pt, ->] (try.south) -- (finally.north);
        \draw[white, ->] (finally.north) -- (finally.south);
        \draw[->] (finally.south) -- +(0,-0.25);
    \end{tikzpicture}
\end{minipage}
\begin{minipage}[t]{0.33\columnwidth}
    \subsubsection{Behandelte Exception}
    \begin{tikzpicture}[>={Stealth[length=1.25mm]}]
        \begin{scope}[every node/.style={draw, 
            fill=bluecontrast, 
            text=white, 
            text width=1.6cm, 
            minimum height=3mm,
            inner sep=0mm, 
            text depth=3mm,
            text height=2mm,
            align=flush left}]
            \node at (0,-0.5) (try) {try};
            \node[fill=redcontrast,anchor=north west, postaction={draw,thin}, line width=0mm] at ($(try.south)!0.5!(try.south east)$) (catch) {catch};
            \node[fill=basegreen, anchor=north east, postaction={draw,thin}, line width=0mm] at ($(catch.south)!0.5!(catch.south west)$) (finally) {finally};
        \end{scope}
        \node[fill=claretcontrast, star, star point height=1.235mm, minimum size=4mm, inner sep=0mm, star points=5, draw=lightgray, anchor=inner point 2, xshift=0.7mm] (someexcept) at (try.center) {};
        \draw[lightgray, dash pattern=on 1.5pt off 1.5pt, ->, looseness=0.5] (someexcept) to[out=0,in=90] (catch.north);
        \draw[->] (0,0) -- (try.north);
        \draw[white, ->] (try.north) -- (try.north |- someexcept.inner point 1);
        \draw[white, ->] (catch.north) -- (catch.south);
        \draw[white, ->] (finally.north) -- (finally.south);
        \draw[->] (finally.south) -- +(0,-0.25);
    \end{tikzpicture}
\end{minipage}\hfill%
\begin{minipage}[t]{0.33\columnwidth}
    \subsubsection{Unbehandelte Exception}
    \begin{tikzpicture}[>={Stealth[length=1.25mm]}]
        \begin{scope}[every node/.style={draw, 
            fill=bluecontrast, 
            text=white, 
            text width=1.6cm, 
            minimum height=3mm,
            inner sep=0mm, 
            text depth=3mm,
            text height=2mm,
            align=flush left}]
            \node at (0,-0.5) (try) {try};
            \node[fill=redcontrast,anchor=north west, postaction={draw,thin}, line width=0mm] at ($(try.south)!0.5!(try.south east)$) (catch) {catch};
            \node[fill=basegreen, anchor=north east, postaction={draw,thin}, line width=0mm] at ($(catch.south)!0.5!(catch.south west)$) (finally) {finally};
        \end{scope}
        \node[fill=claretcontrast, star, star point height=1.235mm, minimum size=4mm, inner sep=0mm, star points=5, draw=lightgray, anchor=inner point 2, xshift=0.7mm] (someexcept) at (try.center) {};
        \draw[lightgray, dash pattern=on 1.5pt off 1.5pt, ->] (someexcept) to[out=-150,in=150] (finally.north);
        \draw[->] (0,0) -- (try.north);
        \draw[white, ->] (try.north) -- (try.north |- someexcept.inner point 1);
        \draw[white, ->] (finally.north) -- (finally.south);
        \node[fill=claretcontrast, star, star point height=1.235mm, minimum size=4mm, inner sep=0mm, star points=5, draw=lightgray, anchor=inner point 2, xshift=0.7mm, yshift=-0.88mm] at (finally.south) {};
    \end{tikzpicture}
\end{minipage}
\lstinline{finally}-Block wird immer ausgeführt, auch wenn \lstinline{catch}-Block eine Exception wirft.

\vspace{1ex}
\begin{minipage}[t]{0.6\columnwidth}
    \subsection{try-with-resources}
    \begin{itemize}
        \item Objekte, die geschlossen werden müssen
        \item \lstinline{AutoCloseable}-Interface
    \end{itemize}
    \lstinputlisting{snippets/excepttrywith.java}
    \subsection{Benutzerdefinierte Exceptions}
    \lstinputlisting{snippets/exceptcustom.java}
\end{minipage}\hfill%
\begin{minipage}[t]{0.39\columnwidth}
    \subsection{Throwable}
    \raggedright%
    \begin{itemize}
        \item Klasse oder Unterklasse von \lstinline{Throwable}
        \item Klassifizierung des Fehlers
    \end{itemize}
    \begin{tikzpicture}[baseline=(current bounding box.north), 
        every node/.style={
            draw, 
            fill=basegreen,
            text=white,
            minimum width=1.5cm,
            minimum height=4.5mm,
            align=center},
        >={Stealth}]

        \node[](throwable) at (0,0) {Throwable};
        \node[below left=2mm and -6mm of throwable](error) {Error};
        \node[below right=2mm and -6mm of throwable](exception) {Exception};
        \node[below=3mm of exception] (runtime) {RuntimeException};
        \draw[->] (error) -- (throwable);
        \draw[->] (exception) -- (throwable);
        \draw[->] (runtime) -- (exception);
        \draw[lightgray, dash pattern=on 1.5pt off 1.5pt, -] (runtime.south) -- +(0,-2mm);  
    \end{tikzpicture}
\end{minipage}
