\section{Unit Tests}

\begin{minipage}[t]{0.58\columnwidth}
    \subsection{Konzept}
    \raggedright%
    \begin{itemize}
        \item Test von abgenzbarem Programmteil (Unit)
        \item Regressionstest: hat eine Änderung bestehende Funktionen geschädigt?
    \end{itemize}
\end{minipage}\hfill%
\begin{minipage}[t]{0.26\columnwidth}
    \subsection{Blackbox-Test}
    \raggedright%
    Testfälle aus Anforderungs- und Schnittstellenbeschr. herleiten
\end{minipage}
\begin{minipage}[t]{0.14\columnwidth}
    \begin{tikzpicture}[baseline=(current bounding box.north), >=Stealth]
        \node[minimum height=5mm, fill=black, text=white] (bbox) at (0,0) {Code};
        \draw[<-] (bbox.north) -- ++(0,0.5) node[midway, right=-0.7mm] {Input};
        \draw[->] (bbox.south) -- ++(0,-0.5) node[midway, right=-0.7mm] {Output};
    \end{tikzpicture}
\end{minipage}
\vspace{-0.5\abovedisplayskip}
\begin{center}
    \begin{tikzpicture}[every node/.style={draw=black, 
        fill=basegreen, 
        text=white, 
        text width=1.3cm, 
        rounded corners=0.2cm,
        minimum height=1cm,
        align=center},
        >=Stealth]
        \node (1) at (0,0) {Ausgans- zustand herstellen};
        \node[right=5mm of 1] (2) {Test ausführen};
        \node[right=5mm of 2] (3) {Soll-Ist Vergleich};
        \node[right=5mm of 3] (4) {Ergebnis dokumentieren};
        \draw[->] (1) -- (2);
        \draw[->] (2) -- (3);
        \draw[->] (3) -- (4);
        \begin{scope}[every node/.style={text=black, minimum height=0, fill=none, draw=none}]
            \node[below=0mm of 1] {Variablen-Deklaration};
            \node[below=0mm of 3] {Assert};
            \node[below=0mm of 4] {Logging};
        \end{scope}
    \end{tikzpicture}
\end{center}
\vspace{-1.3\abovedisplayskip}


\subsection{Äquivalenzklassenbildung}
\vspace{-0.8\abovedisplayskip}
\begin{minipage}[t]{0.5\columnwidth}
    \subsubsection{1. Klassen bilden}
    \raggedright%
    Wertebereich der Parameter in Bereiche zerlegen, die von der Funktion wahrscheinlich gleich behandelt werden.
\end{minipage}\hfill%
\begin{minipage}[t]{0.49\columnwidth}
    \subsubsection{2. Tests erstellen}
    \raggedright%
    Pro Äquivalenzklasse Belegung der Eingangsvariablen wählen und Testfall schreiben.
\end{minipage}

\subsection{Aufbau Testmethode}
\begin{minipage}[t]{0.65\columnwidth}
    \vspace{-0.8\abovedisplayskip}
    \lstinputlisting{snippets/archunit.java}
\end{minipage}\hfill%
\begin{minipage}[t]{0.34\columnwidth}
    \raggedright%
    \begin{itemize}
        \item Keine Parameter
        \item Rückgabetyp \lstinline{void}
        \item \lstinline{¦¦@Test}-Annotation
        \item Asserts um Werte zu prüfen
        \item Testmethoden isoliert von anderen Testmethoden
    \end{itemize}
\end{minipage}

\subsection{Asserts}
\begin{minipage}[t]{0.54\columnwidth}
    \subsubsection{Prüfen von Gleichheit (inhaltlich)}
    \begin{itemize}
        \item \lstinline{assertEquals(expected, actual)}
        \item \lstinline{assertArrayEquals(expected, actual)}
        \item \ldots
    \end{itemize}
\end{minipage}\hfill%
\begin{minipage}[t]{0.45\columnwidth}
    \subsubsection{Prüfen von boolschen Ausdrücken}
    \begin{itemize}
        \item \lstinline{assertTrue(actual)}
        \item \lstinline{assertFalse(actual)}
        \item \ldots
    \end{itemize}
\end{minipage}

\subsection{FIRST-Prinzip}
\begin{tabular}{@{\hspace{1.3mm}}l l@{}}
    \tabitem\textbf{\cgn{F}}ast: &Ausführung ist schnell\\
    \tabitem\textbf{\cgn{I}}ndependent: &Reihenfolge der Tests ist nicht relevant.\\
    \tabitem\textbf{\cgn{R}}epeatable: &Ergebnis ändert sich nur wenn sich Implementierung ändert.\\
    \tabitem\textbf{\cgn{S}}elf-validating: &Testergebnis benötigt keine Interpretation.\\
    \tabitem\textbf{\cgn{T}}imely: &Tests werden früh geschrieben.
\end{tabular}
