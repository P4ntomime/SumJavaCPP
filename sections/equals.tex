\section{Equals-Methode}
\begin{itemize}
    \item Standardmässig vergleicht \lstinline{equals()} auf Referenzgleichheit.
    \item Für Inhaltsvergleich muss \lstinline{equals()} überschrieben werden.
    \begin{itemize}
        \item Nur bei \lstinline{String} bereits implementiert!
    \end{itemize}
    \item Gibt \lstinline{true} zurück, wenn die Objekte gleich sind.
    \item Wird \lstinline{equals()} überschrieben, muss auch \lstinline{hashCode()} überschrieben werden.
\end{itemize}


\begin{minipage}[t]{0.54\columnwidth}
    \subsection{Syntax}
    
    \lstinputlisting{snippets/equals.java}
\end{minipage}\hfill%
\begin{minipage}[t]{0.45\columnwidth}
    \subsection{Vergleiche}
    \begin{itemize}
        \item \lstinline{==} vergleicht Referenzen.
        \item \lstinline{equals()} vergleicht Inhalte.
        \item \lstinline{instanceof} vergleicht Typen.
        \item \lstinline{Objects.equals()} vergleicht Inhalte und behandelt \lstinline{null} richtig.
    \end{itemize}
\end{minipage}
