\section{Input/Output}
% DONE: rename to Input/Output
% DONE: Byte Streams
% DONE: Character Stream
% DONE: File I/O
    % DONE: Zeilenweise lesen
% TODO: Wichtige Methoden
% DONE: Standard I/O
    % DONE: Zeilenweise lesen 
% TODO: Klassenhierarchie
% TODO: Exceptions
% TODO: "einfachste Zugriffe"

\subsection{Streams allgemein}
\vspace{-0.7\abovedisplayskip}
\begin{minipage}[t]{0.5\columnwidth}
    \subsubsection{Input Stream}
    \begin{itemize}
        \item Daten \textbf{von aussen} lesen
        \item Tastatur, Netzwerk, Dateien, \ldots
    \end{itemize}
\end{minipage}\hfill%
\begin{minipage}[t]{0.49\columnwidth}
    \subsubsection{Output Stream}
    \begin{itemize}
        \item Daten \textbf{nach aussen} schreiben
        \item Bildschirm, Netzwerk, Dateien, \ldots
    \end{itemize}
\end{minipage}

\begin{minipage}[t]{0.5\columnwidth}
    \subsection{Byte Streams}
    \raggedright%
    \begin{itemize}
        \item Byteweises lesen (8-Bit Daten)
        \item Erben von: \lstinline{InputStream}, \lstinline{OutputStream}
        \item Beispiel: \lstinline{FileInputStream}, \lstinline{FileOutputStream}
        \item Close bei Exceptions: \lstinline{in.close()} in \lstinline{finally}-Block
    \end{itemize}
\end{minipage}\hfill%
\begin{minipage}[t]{0.49\columnwidth}
    \subsection{Character Streams}
    \raggedright%
    \begin{itemize}
        \item Unicode mit 16-Bit (UTF-16) codiert
        \item Erben von: \lstinline{Reader}, \lstinline{Writer}
        \item Zeichen-/Zeilenweise Ein- \& Ausgabe
    \end{itemize}
\end{minipage}


\subsubsection{\textsf{InputStream}}
\lstinputlisting{snippets/inputstream.java}
\begin{itemize}
    \item Lese \lstinline{length} Bytes in Array \lstinline{b} ab Index \lstinline{offset}
    \item Rückgabe = gelesene Bytes (-1 \textlrarrow{} end of stream)
\end{itemize}

\subsubsection{\textsf{OutputStream}}
\lstinputlisting{snippets/outputstream.java}
\begin{itemize}
    \item Schreibt eventuell noch im Cache zwischengespeicherte Ausgaben
    \item Implizit bei \lstinline{close()}
\end{itemize}

\subsection{File-Input}
% \subsubsection{Beispiel}
\vspace{-0.7\abovedisplayskip}
\begin{minipage}[t]{0.6\columnwidth}
    \lstinputlisting[escapechar=!]{snippets/fileinput.java}
\end{minipage}\hfill%
\begin{minipage}[t]{0.39\columnwidth}
    \begin{tikzpicture}[
        every node/.style={
            align=center,
            minimum width=2cm,
            minimum height=8mm,
            draw,
            rounded corners=2mm},
        every path/.style={
            rounded corners=2pt},
        remember picture, 
        overlay,
        >=Stealth]
        \node[text width=2.2cm, right=1.7cm of fileinnew] (fintxt) {Bestehende Datei zum Lesen öffnen};
        \node[below=2mm of fintxt] (eof) {-1: end of file};
        \node[text width=2cm, below=2mm of eof] (read) {Gelesenes Byte (wenn positiv)};
        \draw[->] (fintxt.west) -- (fileinnew.east);
        \draw[->] (eof.west) -- ++(-1cm,0) |- (fileineof.east);
        \draw[->] (read.west) -- ++(-1.5cm,0) |- (fileinread.east);
    \end{tikzpicture}
\end{minipage}

\subsection{File-Output}
% \subsubsection{Beispiel}
\vspace{-0.7\abovedisplayskip}
\begin{minipage}[t]{0.6\columnwidth}
    \lstinputlisting[escapechar=!]{snippets/fileoutput.java}
\end{minipage}\hfill%
\begin{minipage}[t]{0.39\columnwidth}
    \begin{tikzpicture}[
        every node/.style={
            align=center,
            minimum width=2cm,
            minimum height=8mm,
            draw,
            rounded corners=2mm},
        every path/.style={
            rounded corners=2pt},
        remember picture, 
        overlay,
        >=Stealth]
        \node[text width=2.2cm] (fouttxt) at (fileoutnew.east -| fintxt.south) {Datei neu andlegen bzw. überschreiben};
        \node[text width=2cm] (foutappend) at (fileoutappend.east -| fouttxt.south) {An Datei anhängen, falls existiert};
        \node[text width=3cm] (foutclose) at ($(fouttxt.south)!0.5!(foutappend.north)$) {Schreiben des Rests beim Schliessen ("Flush")};
        \draw[->] (fouttxt.west) -- (fileoutnew.east);
        \draw[->] (foutclose.west) -- ++(-1cm,0) |- (fileoutclose.east);
        \draw[->] (foutappend.west) -- (fileoutappend.east);
    \end{tikzpicture}
\end{minipage}


\subsection{Einfachster Dateizugriff}

\begin{minipage}[t]{0.67\columnwidth}
    Ganze Datei binär einlesen:
    \lstinputlisting[escapechar=!]{snippets/fileinput2.java}
\end{minipage}\hfill%
\begin{minipage}[t]{0.32\columnwidth}
    \begin{tikzpicture}[
        every node/.style={
            align=center,
            minimum width=2.5cm,
            text width=2.5cm,
            minimum height=8mm,
            draw,
            rounded corners=2mm},
        every path/.style={
            rounded corners=2pt},
        remember picture, 
        overlay,
        >=Stealth]
        \node (readall) at ($(fileinreadall.south -| fintxt.south)+(0,-3mm)$) {Speicherintensiv bei grossen Dateien};
        \draw[->] (readall.west) -| (fileinreadall.south);
    \end{tikzpicture}
\end{minipage}
    
\vspace{3mm}
\begin{minipage}[t]{0.48\columnwidth}
    Ganze Datei binär schreiben:
    \lstinputlisting{snippets/fileoutput2.java}
\end{minipage}

\subsection{Standard I/O}
\vspace{-0.7\abovedisplayskip}
\begin{minipage}[t]{0.4\columnwidth}
    \subsubsection{\textsf{System.in}}
    \begin{itemize}
        \item \lstinline{InputStream}
    \end{itemize}
\end{minipage}\hfill%
\begin{minipage}[t]{0.59\columnwidth}
    \subsubsection{\textsf{System.out} und \textsf{System.err}}
    \begin{itemize}
        \item \lstinline{PrintStream} (Subklasse von \lstinline{OutputStream})
    \end{itemize}
\end{minipage}

\subsection{File-Reader}
\begin{minipage}[t]{0.55\columnwidth}
    \vspace{-0.8\abovedisplayskip}
    \lstinputlisting{snippets/filereader.java}
\end{minipage}
\begin{minipage}[t]{0.44\columnwidth}
    \raggedright%
    \begin{itemize}
        \item Systemabhängiges Character Set
        \item \lstinline{val = -1} \textrightarrow{} end of file
        \item value ist 16-Bit Unicode char
    \end{itemize}
\end{minipage}
\subsection{File-Writer}
\begin{minipage}[t]{0.6\columnwidth}
    \vspace{-0.8\abovedisplayskip}
    \lstinputlisting{snippets/filewriter.java}
\end{minipage}
\begin{minipage}[t]{0.39\columnwidth}
    \raggedright%
    \begin{itemize}
        \item \lstinline{true} \textrightarrow{} append
        \item \lstinline{.write("")} String schreiben
        \item \lstinline{.write('')} Einzelnen char schreiben
    \end{itemize}
\end{minipage}

\subsection{Zeilenweise lesen (\textcolor{coralcontrast}{\textsf{InputStream}})}
\begin{tikzpicture}[
    every node/.style={
    align=center,
    minimum width=8mm,
    minimum height=1cm,
    draw,
    rounded corners=2mm},
    >=Stealth]
    \node[text=basegreen, draw=basegreen] (br1) at (0,0) {BufferedReader};
    \node[text width=2.3cm, right=0.5cm of br1] (fr1) {FileReader\\ \textcolor{coralcontrast}{InputStreamReader}};
    \node[draw=coralcontrast, text=coralcontrast, right=0.5cm of fr1] (fis1) {FileInputStream};
    \node[rounded corners=0mm, right=0.5cm of fis1] (f1) {File};
    \begin{scope}[
        every node/.style={
            text depth=0pt, 
            anchor=base,
            draw=none}]
        \node[text=basegreen, below=0mm of br1] (brst) {Zeilenweiser Buffer};
        \node[below=0mm of fr1] (frst) {Character Stream};
        \node[text=coralcontrast, below=0mm of fis1] (fisst) {Byte Stream};
    \end{scope}
    \draw[->] (br1) -- (fr1);
    \draw[->] (fr1) -- (fis1);
    \draw[->] (fis1) -- (f1);
\end{tikzpicture}

% \vfill\null%