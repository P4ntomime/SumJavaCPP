\section{Schnittstellen (Interfaces)}


\subsection{Syntax}
\begin{minipage}[t]{0.51\columnwidth}
    \raggedright%
    \begin{itemize}
        \item Implizit \lstinline{public} und \lstinline{abstract}. (Andere Modifikatoren \textbf{nicht} erlaubt)
        \item Alle Methoden müssen von der implementierenden Klasse implementiert werden, sofern diese nicht \lstinline{abstract} ist.
        \item Methoden dürfen \textbf{keine} Implementierung im Interface haben.
        \item Interfaces können \textbf{nicht} instanziiert werden.
        \item Eine Implementierung kann mehrer Interfaces implementieren.
        \item Falls mehrere Methoden mit gleichem Namen existieren, wird nur eine Methode implementiert.
        \item Konstanten werden automatisch als \lstinline{public static final} deklariert.
    \end{itemize}
\end{minipage}\hfill%
\begin{minipage}[t]{0.48\columnwidth}
    \vspace{-0.8\abovedisplayskip}
    \lstinputlisting{snippets/interface1.java}
    \lstinputlisting{snippets/interface2.java}
\end{minipage}

% ugly workaround to use less space
% FIXME: find a better solution
\begin{itemize}
    \item Falls mehrere Konstanten mit demselben Namen existieren, muss der Name des Interfaces vorangestellt werden.
    \item Schnittstellen können andere Schnittstellen erweitern mit \lstinline{extends}.
\end{itemize}


\subsection{Lose Kopplung}
Mittels loser Kopplung können mehrere Teams einfach miteinander arbeiten. 
Die Teams müssen sich nur auf die Schnittstelle einigen. 
Die Implementierung kann unabhängig voneinander erfolgen.

\begin{minipage}[t]{0.54\columnwidth}
    \subsection{Abstrakte Klassen}
    Klasse, die nicht instanziiert werden kann und mindestens eine abstrakte Methode enthält.
\end{minipage}\hfill%
\begin{minipage}[t]{0.45\columnwidth}
    \subsection{Abstrakte Methoden}
    Deklaration einer Methode ohne Implementierung. Geht nur in abstrakten Klassen und Interfaces.
    \lstinputlisting{snippets/abstractmethod.java}
\end{minipage}

\vspace{-7mm}
\subsection{Set-Interface}
Menge von Elementen ohne Duplikate.
\subsubsection{Methoden}
\begin{itemize}
    \item \lstinline{boolean add(E e)}: Fügt Element hinzu, wenn nicht bereits vorhanden
    \item Gleichheitsprüfung mit \lstinline{equals()}
\end{itemize}

\subsubsection{Beispiel}
\lstinputlisting{snippets/set.java}
\subsection{Map-Interface}
\begin{itemize}
    \item Objekt, das Schlüssel auf Werte abbildet.
    \item Kann keine Duplikate von Schlüsseln enthalten.
    \item Jeder Schlüssel kann höchstens einem Wert zugeordnet werden.
\end{itemize}
\begin{minipage}[t]{0.31\columnwidth}
    \subsubsection{Visualisierung}
    \begin{tikzpicture}[every node/.style={align=center}, >=Stealth]
        \def\deltay{1.5mm}
        \def\minheight{8mm}
        \begin{scope}[every node/.style={text=redcontrast, minimum height=\minheight}]
            \node[minimum height=0.2cm] (keyname) at (0,0) {Matrikelnr.};
            \node[below=1mm of keyname] (key1) {20000};
            \node[below=\deltay of key1] (key2) {70000};
            \node[below=\deltay of key2] (key3) {13000};
        \end{scope}
        \begin{scope}[every node/.style={align=center,fill=basegreen, text=white, draw, rounded corners=2mm, minimum width=1.3cm, minimum height=\minheight}]
            \node[minimum height=0.2cm,fill=none,draw=none,text=basegreen,right=0mm of keyname] (valuename) {Student};
            \node[below=1mm of valuename] (value1) {Hans\\ Kuster};
            \node[below=\deltay of value1] (value2) {Lars\\ Peter};
            \node[below=\deltay of value2] (value3) {Anna\\ Meier};
        \end{scope}
        \draw[->] (key1.east) -- (value1.west);
        \draw[->] (key2.east) -- (value2.west);
        \draw[->] (key3.east) -- (value3.west);
    \end{tikzpicture}    
\end{minipage}\hfill
\begin{minipage}[t]{0.69\columnwidth}
    \subsubsection{Beispiel}
    \phantom{t}\vspace{-3mm}
    \lstinputlisting[morekeywords={Integer}, inputencoding=cp1252]{snippets/map.java} % map.java needs to be saved with Windows 1252 (cp1252) encoding!
\end{minipage}

\subsection{Comparator-Interface}
\vspace{-0.8\abovedisplayskip}
\begin{minipage}[t]{0.45\columnwidth}
    \lstinputlisting{snippets/compintf.java}
\end{minipage}
\begin{minipage}[t]{0.27\columnwidth}
    \subsubsection{Returnwert \textnormal{\textsf{\textcolor{black}{compare}}}}
    \begin{itemize}
        \item \textbf{< 0}: \lstinline{o1} < \lstinline{o2}
        \item \textbf{\phantom{< }0}: \lstinline{o1} = \lstinline{o2}
        \item \textbf{> 0}: \lstinline{o1} > \lstinline{o2}
    \end{itemize}
\end{minipage}\hfill%
\begin{minipage}[t]{0.26\columnwidth}
    \subsubsection{Returnwert \textnormal{\textsf{\textcolor{black}{equals}}}}
    \begin{itemize}
        \item \lstinline{true}:\phantom{e} \lstinline{obj} = \lstinline{this}
        \item \lstinline{false}: \lstinline{obj} \tikz{\node[inner sep=0mm](a) at (0,0){=}; \draw[] ($(a.south west)!0.5!(a.south)+(0,0.1mm)$) -- ($(a.north east)!0.5!(a.north)+(0,0.1mm)$);} \lstinline{this}
    \end{itemize}
\end{minipage}

\subsection{Functional Interface}
\begin{itemize}
    \item Interface mit genau einer abstrakten Methode
    \item Kann mit Lambda-Ausdrücken verwendet werden
    \item Annotation \lstinline{¦¦@FunctionalInterface} ist optional aber empfohlen
    \item Methodenreferenzen passen auf funktionale Interfaces
\end{itemize}
\subsubsection{Beispiel}

\begin{minipage}[t]{0.42\columnwidth}
    \myul{Funktionales Interface:}
    \lstinputlisting{snippets/functionalintf1.java}
\end{minipage}\hfill%
\begin{minipage}[t]{0.57\columnwidth}
    \myul{Implementierung:}
    \lstinputlisting{snippets/functionalintf2.java}
\end{minipage}
\myul{Methodenreferenz:}
\lstinputlisting{snippets/functionalintf3.java}

\vfill\null%

% DONE: Comparator Interface
% DONE: Functional Interfaces
