\section{Klassen}
\subsection{Definition}
Eine Klasse spezifiziert die Gemeinsamkeiten einer Menge von Objekten mit denselben Eigenschaften, 
demselben Verhalten und denselben Beziehungen. [Balzert]

\begin{minipage}[t]{0.5\columnwidth}
    \subsection{Aufbau}
    % \vspace{-2mm}
    \lstinputlisting{snippets/klasse.java}
\end{minipage}\hfill%
\begin{minipage}[t]{0.49\columnwidth}
    \subsubsection{Konstruktor}
    \raggedright%
    \begin{itemize}
        \item Initialisiert ein Objekt/eine Klasse
        \item Hat keinen Rückgabewert
        \item Hat gleichen Namen wie Klasse
        \item Kann überladen werden
        \item Compiler erzeugt einen Standardkonstruktor, wenn kein Konstruktor definiert ist
    \end{itemize}
\end{minipage}

\begin{minipage}[t]{0.41\columnwidth}
    \subsection{\textsf{\textbf{\textcolor{keywordcolour}{null}}}-Referenz}
    \raggedright%
    \begin{itemize}
        \item Referenz auf "kein Objekt"
        \item Meist zur Initialisierung von Referenzen verwendet
        \item Gültig für alle Referenztypen
        \item Dereferenzierung führt zu \lstinline{NullPointerException}
    \end{itemize}
\end{minipage}\hfill%
\begin{minipage}[t]{0.59\columnwidth}
    \raggedright%
    \subsection{Selbstreferenz}
    Zur Selbstreferenz wird das Schlüsselwort \lstinline{this} verwendet.
    
    \subsection{Instanziierung}
    Objekte werden mit dem \lstinline{new}-Operator erzeugt (instanziiert): \mylstbox{Rectangle r = new Rectangle();}
\end{minipage}

