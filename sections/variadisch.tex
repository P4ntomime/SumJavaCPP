\section{Variadische Methoden}
Variadische Methoden sind Methoden, die eine variable Anzahl an Argumenten akzeptieren.
\subsection{Syntax}
\vspace{-0.8\abovedisplayskip}
\begin{minipage}[t]{0.49\columnwidth}
    \subsubsection{Definition}
    \lstinputlisting{snippets/variadisch.java}
\end{minipage}\hfill
\begin{minipage}[t]{0.49\columnwidth}
    \subsubsection{Aufruf}
    \lstinputlisting{snippets/variadisch2.java}
\end{minipage}
\begin{itemize}
    \item Compiler erzeugt für variable Parameter ein Array, welches Argumente enthält.
    \item Argumente können jeweils nur von \textbf{einem} Typ sein.
    \item Parameter in der Variadischen Funktion muss \textbf{immer} der letzte Parameter sein.
    \item Parameter kann auch als Array übergeben werden.
\end{itemize}