\section{Java Virtual Machine}
\subsection{Java Runtime-Architektur}

\begin{tikzpicture}[
    every node/.style={
        draw=black, 
        minimum height=0.8cm,
        minimum width=1.15cm,
        text width=1.15cm,
        align=center},
    >=Stealth]
    \node (srccode) {Quellcode};
    \node[fill=basegreen, text=white, right=0.4cm of srccode, rounded corners=0.2cm] (javac) {Java Compiler};
    \node[right=0.9cm of javac] (bytecode) {Bytecode};
    \node[fill=basegreen, text=white, right=0.55cm of bytecode, rounded corners=0.2cm] (jvm) {Java Virtual Machine};
    \node[fill=redcontrast, text=white, right=0.55cm of jvm, rounded corners=0.2cm] (os) {Reale Maschine};

    \begin{scope}[
        every node/.style={
            draw=none, 
            minimum height=0.3cm, 
            minimum width=0.3cm, 
            align=center, 
            text width=0.9cm,
            font=\small}]
        \draw[->] (srccode) -- (javac);
        \draw[->] (javac.east) to node[midway, above] {erzeugt} (bytecode.west);
        \draw[->] (bytecode) to node[midway, above] {läuft\\ auf} (jvm);
        \draw[->] (jvm) to node[midway, above] {läuft\\ auf} (os);
    \end{scope}
\end{tikzpicture}

\subsection{Bytecode}
\begin{minipage}{0.25\columnwidth}
    \lstinputlisting[style=basestyle, xleftmargin=1em, numbers=left, escapechar=!]{snippets/bytecode.java}
\end{minipage}\hfill
\begin{minipage}{0.71\columnwidth}
    \begin{tikzpicture}[every node/.style={draw=none, align=flush left, text width=2.1cm, minimum height=0.1cm}]
        \node[text=claretcontrast] (0) {\texttt{BIPUSH 12}\\\vspace{-1mm}\texttt{ISTORE 1}};
        \node[right=0cm of 0, text=claretcontrast, text width=4.2cm] {Wert 12 auf Stack legen\\Wert von Stack an Index 1 speichern};

        \node[below=0.1cm of 0, text=basegreen] (1) {\texttt{ICONST\_1}\\\vspace{-1mm}\texttt{ISTORE 2}};
        \node[right=0cm of 1, text=basegreen, text width=4.2cm] {Wert 11 auf Stack legen\\Wert von Stack an Index 2 speichern};

        \node[below=0.1cm of 1, text=redcontrast] (2) {\texttt{ICONST\_1}\\\vspace{-1mm}\texttt{ISTORE 3}};
        \node[right=0cm of 2, text=redcontrast, text width=4.2cm] {Wert 1 auf Stack legen\\Wert von Stack an Index 3 speichern};
        
        \node[below=0.1cm of 2, text=coralcontrast] (3) {\texttt{ILOAD 3}\\\vspace{-1mm}\texttt{ILOAD 1}};
        \node[below=-1mm of 3, text=coralcontrast] (3a) {\texttt{IF\_ICMPLE L4}};
        \node[right=0cm of 3a, text=coralcontrast, text width=4.2cm] (3acomment) {Wenn 3 > 1};
        \node[above=-2mm of 3acomment, text=coralcontrast, text width=4.2cm] (3b) {Wert Index 3 auf Stack legen\\Wert Index 1 auf Stack legen};
        \node[below=-1mm of 3a, text=lightgray] (4) {\texttt{//\ldots}};
    \end{tikzpicture}
\end{minipage}

\subsubsection{Java Bytecode}
\begin{itemize}
    \item Standardisierte Zwischensprache
    \item Für hypothetische Stack-Maschine
\end{itemize}

\subsubsection{Java Virtual Machine}
\begin{itemize}
    \item Interpretiert Bytecode
    \item Implementiert für jede Zielplattform
    \item Just-In-Time (JIT) Compiler in realen Maschinencode
\end{itemize}