\section{Stream-API}


\begin{minipage}[t]{0.6\columnwidth}
    \begin{itemize}
        \item Für deklarative Verarbeitung von Datenströmen
        \item Definiere \textbf{was} gemacht werden soll, nicht \textbf{wie}
    \end{itemize}
    \subsection{Beispiel}
    \lstinputlisting{snippets/streamapiex.java}
\end{minipage}\hfill%
\begin{minipage}[t]{0.39\columnwidth}
    \begingroup
    % change \section command temporarily to be centered
    \titleformat{\subsection}
            {\fontsize{9}{8}\selectfont\bfseries\filcenter}
            {\thesubsection}
            {0mm}
            {\phantomsection\myul{#1}} % \phantomsection to make hyperref link to correct section
    \subsection{Idee}
    \endgroup
    \begin{center}
        \begin{tikzpicture}[every node/.style={
            draw, 
            text=white, 
            rounded corners=2mm,
            minimum width=2.1cm,
            text width=2.1cm,
            minimum height=7mm, 
            align=center},
            >=Stealth]
            \node[fill=bluecontrast] (a) at (0,0) {Stream erzeugen};
            \node[fill=redcontrast, below=3mm of a] (b) {Zwischen- operationen};
            \node[fill=basegreen, below=3mm of b] (c) {Terminaloperation};
            \draw[->] (a) -- (b);
            \draw[->] (b) -- (c);
        \end{tikzpicture}
    \end{center}
\end{minipage}



\subsection{Endliche Quellen}
\begin{tabular}{@{\hspace{1.3mm}}ll@{}}
    \tabitem\mylstbox{IntStream.range(0, 10)}: &Zahlen von 0 bis 9\\
    \tabitem\mylstbox{Stream.of(2, 3, 4)}: &Eigene Aufzählung\\
    \tabitem\mylstbox{Stream.empty()}: &Leerer Stream\\
    \tabitem\mylstbox{Collection.stream()}: &Stream aus Collection\\
    \tabitem\mylstbox{Stream.concat(s1, s2)}: &Verkettung zweier Streams
\end{tabular}

\subsection{Unendliche Quellen}
\vspace{-0.7\baselineskip}
\begin{minipage}[t]{0.5\columnwidth}
    \subsubsection{\textsf{generate()}}
    \lstinputlisting{snippets/streamgenerate.java}
\end{minipage}\hfill%
\begin{minipage}[t]{0.49\columnwidth}
    \subsubsection{\textsf{iterate()}}
    \lstinputlisting{snippets/streamiterate.java}
\end{minipage}

\subsection{Zwischenoperationen}
\begin{minipage}[t]{\columnwidth}
    \raggedright%
    Bei Collections ist es \textbf{nicht} erlaubt, diese mit Zwischenoperationen zu ändern. Auch nicht erlaubt sind Abhängigkeiten zu äusseren, änderbaren Variablen.
\end{minipage}
\begin{tabular}{@{\hspace{1.3mm}}l@{\hspace{1mm}}l@{}}
    \tabitem\mylstbox{filter(Predicate)}: &Filtern mit Predicate-Funktionsobjekt/Lambda\\
    \tabitem\mylstbox{map(Function)}: &Projizieren mit Funktionsobjekt/Lamda\\
    \tabitem\mylstbox{mapToInt...(Function)}: &Projizieren auf \lstinline|int|, \lstinline|long|, \lstinline|double|\\
    \tabitem\mylstbox{sorted()}: &Sortieren mit/ohne Comparator\\
    \tabitem\mylstbox{distinct()}: &Duplikate entfernen (\lstinline|equals()|)\\
    \tabitem\mylstbox{limit(long n)}: &n-Elemente liefern\\
    \tabitem\mylstbox{skip(long n)}: &n-Elemente überspringen
\end{tabular}
\vfill\null%

\subsection{Terminaloperationen}
\begin{tabular}{@{\hspace{1.3mm}}l@{\hspace{1mm}}l@{}}
    \tabitem\mylstbox{forEach(Consumer)}: &Pro Element Operation anwenden, meist mit Seiteneffekt\\
    \tabitem\mylstbox{count()}: &Anzahl Elemente\\
    \tabitem\mylstbox{min()}, \mylstbox{max()}: &Mit Comparator-Argument\\
    \tabitem\mylstbox{average()}, \mylstbox{sum()}: &Nur für numerische Streams\\
    \tabitem\mylstbox{findAny()}, \mylstbox{findFirst()}: &Gibt irgendein / erstes Element zurück
\end{tabular}



\begin{minipage}[t]{0.49\columnwidth}
    \subsection{Optional-Wrapper}
    \raggedright%
    \begin{itemize}
        \item Wert exisitiert oder nicht
        \item \lstinline{average()}, \lstinline{min()}, \lstinline{max()} geben \lstinline{Optional} zurück
        \item Überprüfung mit \lstinline{isPresent()}
    \end{itemize}
\end{minipage}\hfill%
\begin{minipage}[t]{0.49\columnwidth}
    \subsection{Matching}
    \raggedright%
    \begin{itemize}
        \item \lstinline{allMatch()}, \lstinline{anyMatch()}, \lstinline{noneMatch()}
        \item Prüfen ob Prädikat auf alle/irgendein/kein Element zutrifft
    \end{itemize}
\end{minipage}


\subsection{Lazy Evaluation}
\begin{itemize}
    \item Element wird erst bereitgestellt, wenn Nachfolger Element anfordert
    \item Unendliche Streams sind meist Lazy
\end{itemize}


\subsection{Collectors}
\begin{tabular}{@{\hspace{1.3mm}}l@{\hspace{1mm}}l@{}}
    \tabitem\mylstbox{Collectors.toList()}: &Liste aller Elemente\\
    \tabitem\mylstbox{Collectors.toCollection(TreeSet::new)}: &In beliebige Collection abbilden\\
    \tabitem\mylstbox{Collectors.groupingBy(key, aggregator)}: &Gruppierung, Aggregator optional\\
    \multicolumn{2}{l}{\phantom{\tabitem}Aggregator: (\lstinline|averaging|, \lstinline|summing|, \lstinline|counting|)}
\end{tabular}


\subsection{Gruppierungen}
\lstinputlisting[morekeywords={Integer}]{snippets/streamgrouping1.java}
\lstinputlisting[morekeywords={Integer}]{snippets/streamgrouping2.java}