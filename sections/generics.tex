\section{Generics}
Generics werden verwendet um einen einheitlichen Elementtypen zu forcieren. Beispielsweise in einer ArrayList.
% TODO: evtl. erwähnen, dass static nicht nötig ist
\subsection{Benennungskonventionen}
\begin{tabularx}{\columnwidth}{@{}l l l@{}}
    \tabitem{}E -- Element & \tabitem{}N -- Number & \tabitem{}V -- Value\\

    \tabitem{}K -- Key &\tabitem{}T -- Type &\tabitem{}S, U, V, \ldots -- 2nd, 3rd, 4th types
\end{tabularx}

\subsection{Generische Typen}
\vspace{-0.8\abovedisplayskip}
\begin{minipage}[t]{0.6\columnwidth}
    \subsubsection{Verschiedene generische Typen}
    \lstinputlisting[morekeywords={Integer}]{snippets/gentyp1.java}
\end{minipage}\hfill%
\begin{minipage}[t]{0.4\columnwidth}
    \subsubsection{Statische Typ-Prüfung}
    \lstinputlisting{snippets/gentyp2.java}
\end{minipage}

\subsubsection{Kein Type-Cast}
\lstinputlisting{snippets/gentyp3.java}

\subsection{Generische Klasse}
\vspace{-0.7\abovedisplayskip}
\begin{minipage}[t]{0.5\columnwidth}
    \subsubsection{Typ-Parameter}
    Platzhalter für generischen Typ.
    \lstinputlisting[escapechar=!]{snippets/genclass1.java}
    \begin{tikzpicture}[remember picture, overlay]
        \draw[semithick, color=bluecontrast,decoration={brace,mirror,raise=1pt}, decorate] (tp1.south) -- node[text=bluecontrast,below=3pt] {Typ-Parameter} (tp2.south);
    \end{tikzpicture}
\end{minipage}\hfill
\begin{minipage}[t]{0.5\columnwidth}
    \subsubsection{Typ-Argument}
    Typ bei Einsatz angeben.
    \lstinputlisting[style=basestyle,numbers=left,xleftmargin=1.2em,escapechar=!]{snippets/genclass2.java}
    \begin{tikzpicture}[remember picture, overlay]
        \draw[semithick, color=redcontrast,decoration={brace,mirror,raise=1pt}, decorate] (ta1.south) -- node[text=redcontrast,below=3pt] {Typ-Argument} (ta2.south);
    \end{tikzpicture}
\end{minipage}

\begin{itemize}
    \item Typ-Parameter dient als "Typ-Variable" innerhalb der generischen Klasse.
    \item Wie normaler Typ verwendbar.
\end{itemize}

\subsection{Generische Methode}

\begin{minipage}[t]{0.5\columnwidth}
    \vspace{-0.8\abovedisplayskip}
    \lstinputlisting{snippets/genmet1.java}
\end{minipage}\hfill%
\begin{minipage}[t]{0.49\columnwidth}
    \raggedright%
    Beim Aufruf generischer Methoden muss der Typ nicht angegeben werden.
\end{minipage}


\subsection{Generische Interfaces}
\subsubsection{Beispiel mit Iterator}
\begin{minipage}[t]{0.5\columnwidth}
    \vspace{-0.7\abovedisplayskip}
    \lstinputlisting{snippets/genint1.java}
\end{minipage}\hfill%
\begin{minipage}[t]{0.5\columnwidth}
    \vspace{-0.7\abovedisplayskip}
    \lstinputlisting{snippets/genint2.java}
\end{minipage}

\subsection{Iteration}
\begin{minipage}[t]{0.32\columnwidth}
    For-Schleife:
    \lstinputlisting{snippets/genericfor1.java}
\end{minipage}\hfill%
\begin{minipage}[t]{0.68\columnwidth}
    Tatsächlich generierter Code:
    \lstinputlisting{snippets/genericfor2.java}
\end{minipage}

\subsection{Type Bounds}
\vspace{-0.8\abovedisplayskip}
\begin{minipage}[t]{0.39\columnwidth}
    \subsubsection{Nutzen}
    \raggedright%
    \begin{itemize}
        \item Typ-Argument \textbf{muss} Subtyp von \lstinline{Graphic} sein
    \end{itemize}
\end{minipage}\hfill%
\begin{minipage}[t]{0.6\columnwidth}
    \subsubsection{Beispiel}
    \lstinputlisting{snippets/gentb1.java}
    Mehrere Bounds können mit \& verknüpft werden.
\end{minipage}
\begin{itemize}
    \item Bei Verwendung von spez. Funktionen innerh.\ generischer Funktion/Klasse
\end{itemize}


% \vfill\null%

% DONE: Generische Methoden
    % DONE: Type Inferenz bei generischen Methoden
% DONE: Generische Interfaces
% DONE: Type Bounds von Parametern