\section{ArrayList}
\subsection{Eigenschaften}
\begin{itemize}
    \item ArrayList enthält Referenzen auf Objekte
    \item Kann dynamisch vergrössert/verkleinert werden
    \item Kann nicht direkt mit primitiven Datentypen (\lstinline{int}, \lstinline{float}, etc.) verwendet werden
\end{itemize}

\subsection{Syntax}
\mylstbox{var stringList = new ArrayList<String>();}

\subsection{Wrapping}
Um einen Primitiven Datentyp zu referenzieren, muss dieser zuerst in ein Objekt gewrappt werden.
Boxing/Unboxing implizit möglich.
\vspace{-0.8\abovedisplayskip}
\begin{center}
    \begin{tabularx}{0.45\columnwidth}{@{}l l@{}}
        \textbf{Primitiver Typ} & \textbf{Wrapper-Klasse}\\\hhline{==}
        \lstinline{boolean} & \lstinline{Boolean}\\\hhline{--}
        \lstinline{char} & \lstinline{Character}\\\hhline{--}
        \lstinline{byte} & \lstinline{Byte}\\\hhline{--}
        \lstinline{short} & \lstinline{Short}\\
    \end{tabularx}
    \begin{tabularx}{0.45\columnwidth}{@{}l l@{}}
        \textbf{Primitiver Typ} & \textbf{Wrapper-Klasse}\\\hhline{==}
        \lstinline{int} & \lstinline{Integer}\\\hhline{--}
        \lstinline{long} & \lstinline{Long}\\\hhline{--}
        \lstinline{float} & \lstinline{Float}\\\hhline{--}
        \lstinline{double} & \lstinline{Double}\\
    \end{tabularx}
\end{center}
    
\subsection{Weitere Collections}
\begin{tabular}{@{}l l@{}}
    \lstinline!List! &Folge von Elementen\\
    \lstinline!Set! &Menge von Elementen\\
    \lstinline!Map! &Abbildung von Schlüssel auf Wert
\end{tabular}
